\documentclass[12pt,letterpaper]{article}
\usepackage{fullpage}
\usepackage[top=2cm, bottom=4.5cm, left=2.5cm, right=2.5cm]{geometry}
\usepackage{amsmath,amsthm,amsfonts,amssymb,amscd}
\usepackage{lastpage}
\usepackage{enumerate}
\usepackage{fancyhdr}
\usepackage{mathrsfs}
\usepackage{xcolor}
\usepackage{graphicx}
\usepackage{listings}
\usepackage{hyperref}
\usepackage{enumitem}

\hypersetup{%
  colorlinks=true,
  linkcolor=blue,
  linkbordercolor={0 0 1}
}
 
\renewcommand\lstlistingname{Algorithm}
\renewcommand\lstlistlistingname{Algorithms}
\def\lstlistingautorefname{Alg.}

\lstdefinestyle{Python}{
    language        = Python,
    frame           = lines, 
    basicstyle      = \footnotesize,
    keywordstyle    = \color{blue},
    stringstyle     = \color{green},
    commentstyle    = \color{red}\ttfamily
}

\setlength{\parindent}{0.0in}
\setlength{\parskip}{0.05in}

% Edit these as appropriate
\newcommand\course{10 - 725}
\newcommand\hwnumber{2}                  % <-- homework number
\newcommand\NetIDa{abc}           % <-- NetID of person #1

\pagestyle{fancyplain}
\headheight 35pt
\lhead{\NetIDa}
\lhead{\NetIDa}                 % <-- Comment this line out for problem sets (make sure you are person #1)
\chead{\textbf{\Large Homework \hwnumber}}
\rhead{\course \\ \today}
\lfoot{}
\cfoot{}
\rfoot{\small\thepage}
\headsep 1.5em


% if you want to create a new list from scratch
\newlist{alphalist}{enumerate}{1}
% in that case, at least label must be specified using \setlist
\setlist[alphalist,1]{label=\textbf{\alph*.}}

\usepackage{amsmath}
\DeclareMathOperator*{\argmax}{arg\,max}
\DeclareMathOperator*{\argmin}{arg\,min}

\begin{document}

\section*{1. Subgradients and Proximal Operators}

Answer to the problem goes here.

\begin{alphalist}

%==================================
\item
\begin{enumerate}[label=(\roman*)]
    \item
        Convex: $\forall u, v \in \partial{}f(x), t \in [0, 1]$. For all $y \in
        dom f$:
        \[
            \begin{split}
                & f(x) + (tu + (1-t)v)^T(y-x) \\
                =\quad & f(x) + tu^T(y-x) + (1-t)v^T(y-x) \\
                \le\quad & f(x) + max\{u^T(y-x), v^T(y-x)\} \\
                \le\quad & f(y)
            \end{split}
        \]

        Closed: $\partial{}f(x) = \{ g : g^T(y - x) \le f(y) - f(x) \}$ is a
        polyhedron, hence closed set.

    \item
        Consider $g \in \partial{}f(x)$, then $f(y) \ge f(x) + g^T(y-x),
        \forall x, y \in dom f$. With $f(y) \le f(x)$, $g^T(y - x) = g^T -
        g^Tx$ must be negative.
        Therefore $g \in N_{y: f(y) \le f(x)}$.

    \item
        If $||y||_q = 0$ then $y = 0$. The inequality hold.

        If $||y||_q > 0$ then $\left|\left|\frac{y}{||y||_q}\right|\right|_q =
        1$. By dual representation of the norm:
        \[
            ||x||_p ||y||_q  \ge x^T\frac{y}{||y||_q} ||y||_q = x^Ty
        \]

    \item
        $S = \argmax_{||z||_q \le 1} z^Tx \subseteq \partial{}f(x)$: Let $g \in
        S$, then $||g||_q \le 1$ and $g^Tx = ||x||_p$. For all $y \in dom f$
        \[
            f(x) + g^T(y-x) = ||x||_p + g^Ty - g^Tx = g^Ty \le ||y||_p = f(y)
        \]

        $\partial{}f(x) \subseteq S$: Let g is a subgradient of $f(x)$.

        If $x = 0$, subgradient definition implies $||y||_p \ge g^Ty \quad \forall
        y\in dom f$, then $||g||_q \le 1$.

        If $x \ne 0$, $\partial{}f(x) = \{g\} = \{\nabla{}||x||_p\} = \left(
        \frac{x_i^{p-1}}{||x||_p^{1/q}} \right)^T$. Then $||g||_q =
        \left( \sum_i\frac{x_i^{(p-1)q}}{||x||_p} \right)^{1/q} = 1$.


        So $||g||_q \le 1 \quad \forall x$.


        Substitute $y = 0$ into subgradient definition: $\max_{||z||_q
        \le 1} z^Tx = ||x||_p \le g^Tx \implies g \in \argmax_{||z||_q \le 1}
        z^Tx$.

\end{enumerate}
\end{alphalist}

%%%%%%%%%%%%%%%%%%%%%%%%%%%%%%%%%%%%%%%%%%%%%%%%%%%%%%%%%%%%%%%%%%%%%%%%%%%%%%%%%%%
%\newpage

\end{document}
