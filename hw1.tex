\documentclass[12pt,letterpaper]{article}
\usepackage{fullpage}
\usepackage[top=2cm, bottom=4.5cm, left=2.5cm, right=2.5cm]{geometry}
\usepackage{amsmath,amsthm,amsfonts,amssymb,amscd}
\usepackage{lastpage}
\usepackage{enumerate}
\usepackage{fancyhdr}
\usepackage{mathrsfs}
\usepackage{xcolor}
\usepackage{graphicx}
\usepackage{listings}
\usepackage{hyperref}
\usepackage{enumitem}

\hypersetup{%
  colorlinks=true,
  linkcolor=blue,
  linkbordercolor={0 0 1}
}
 
\renewcommand\lstlistingname{Algorithm}
\renewcommand\lstlistlistingname{Algorithms}
\def\lstlistingautorefname{Alg.}

\lstdefinestyle{Python}{
    language        = Python,
    frame           = lines, 
    basicstyle      = \footnotesize,
    keywordstyle    = \color{blue},
    stringstyle     = \color{green},
    commentstyle    = \color{red}\ttfamily
}

\setlength{\parindent}{0.0in}
\setlength{\parskip}{0.05in}

% Edit these as appropriate
\newcommand\course{10 - 725}
\newcommand\hwnumber{1}                  % <-- homework number
\newcommand\NetIDa{abc}           % <-- NetID of person #1

\pagestyle{fancyplain}
\headheight 35pt
\lhead{\NetIDa}
\lhead{\NetIDa}                 % <-- Comment this line out for problem sets (make sure you are person #1)
\chead{\textbf{\Large Homework \hwnumber}}
\rhead{\course \\ \today}
\lfoot{}
\cfoot{}
\rfoot{\small\thepage}
\headsep 1.5em


% if you want to create a new list from scratch
\newlist{alphalist}{enumerate}{1}
% in that case, at least label must be specified using \setlist
\setlist[alphalist,1]{label=\textbf{\alph*.}}

\begin{document}

\section*{Problem 1}

Answer to the problem goes here.

\begin{alphalist}

\item Closed and convex sets.
\begin{enumerate}[label=(\roman*)]
    \item
        For any $Ax, Ay \in A(S)$ ($x, y \in S$) and any $\theta \in [0, 1]$, we have

        \[\theta Ax + (1 - \theta) Ay = A (\theta x + (1- \theta)y)\]

        is a member of $A(S)$ because $\theta x + (1-\theta) y \in S$.
        Therefore $A(S)$ is convex.

    \item
        For any $x, y \in A^{-1}(S)$, we have $Ax, Ay \in S$.

        Then $\theta Ax + (1-\theta)Ay = A(\theta x + (1-\theta)y) \in S,
        \forall \theta \in [0, 1]$ because $S$ is convex.

        By definition of $A^{-1}(S)$, $\theta x + (1-\theta)y \in A^{-1}(S)$. Therefore $A^{-1}(S)$ is convex.

    \item
        Let $x$ is a limit point of $A^{-1}(S)$, then $\forall{}\epsilon{}>0,
        \exists{}y \in A^{-1}(S): ||x - y|| = ||z|| < \epsilon$. We will show
        that $Ax$ is a limit point of $S$.

        Consider the l2-norm
        \[
        \begin{split}
            ||Ax - Ay|| &= ||Az||  = \left|\left|\begin{matrix} \sum_j A_{1j}z_j \\ \sum_j A_{2j}z_j \\ .. \\ \sum_j A_{mj}z_j\end{matrix}\right|\right|
                        = \sqrt{\sum_i^m\sum_j^nA_{ij}^2 z_j^2} \\
                        &\le \sqrt{\sum_i^m \sum_j^n A_{max}^2 z_j^2} =\sqrt{m} A_{max} ||z|| < \sqrt{m}A_{max}\epsilon \\
                        &\text{with} \quad A_{max} = \max{|A_{ij}|}.
        \end{split}
    \]

        So with point $Ax$, for all $\xi = \sqrt{m}A_{max}\epsilon>0$, we can find
        point $Ay \in S$ (because $y \in A^{-1}(S)$) such that $|| Ax - Ay || < \xi$. By definition, $Ax$
        is a limit point of $S$. Since $S$ is closed set, $Ax$ is also in $S$,
        it follow that $x \in A^{-1}(S)$.

        We have shown that any limit point $x$ of $A^{-1}(S)$ is also an
        element of $A^{-1}(S)$, therefore it is a closed set.

    \item Consider $S = \{(x, y) : x > 0, y \ge \frac{1}{x}\}$, S is a closed
        set because the compliment set $R^2 \backslash S$ equal to its interior (for any
        point, we can find a disc containing the point that doesn't intersect
        $S$).

        Image of S under transformation $A = (\begin{smallmatrix} 1 & 0
        \end{smallmatrix})$ is $A(S) = \{ x : x > 0 \}$, obviously is not a
        closed set.

\end{enumerate} % end of 1a

\end{alphalist}

%\begin{enumerate}
%  \item
%   Problem 1 part 1 answer here.
%  \item
%    Problem 1 part 2 answer here.
%
%    Here is an example typesetting mathematics in \LaTeX
%\begin{equation*}
%    X(m,n) = \left\{\begin{array}{lr}
%        x(n), & \text{for } 0\leq n\leq 1\\
%        \frac{x(n-1)}{2}, & \text{for } 0\leq n\leq 1\\
%        \log_2 \left\lceil n \right\rceil \qquad & \text{for } 0\leq n\leq 1
%        \end{array}\right\} = xy
%\end{equation*}
%
%    \item Problem 1 part 3 answer here.
%
%    Here is an example of how you can typeset algorithms.
%    There are many packages to do this in \LaTeX.
%     
%    \lstset{caption={Caption for code}}
%    \lstset{label={lst:alg1}}
%     \begin{lstlisting}[style = Python]
%    from package import Class # Mesh required for..
%    
%    cinstance = Class.from_obj('class.obj')
%    cinstance.go()
%    \end{lstlisting}
%     
%  \item Problem 1 part 4 answer here.
%
%    Here is an example of how you can insert a figure.
%    %\begin{figure}[!h]
%    %\centering
%    %\includegraphics[width=0.3\linewidth]{heidi.jpg}
%    %\caption{Heidi attacked by a string.}
%    %\end{figure}
%\end{enumerate}


\section*{Problem 2}
% Rest of the work...

\end{document}
