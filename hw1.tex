\documentclass[12pt,letterpaper]{article}
\usepackage{fullpage}
\usepackage[top=2cm, bottom=4.5cm, left=2.5cm, right=2.5cm]{geometry}
\usepackage{amsmath,amsthm,amsfonts,amssymb,amscd}
\usepackage{lastpage}
\usepackage{enumerate}
\usepackage{fancyhdr}
\usepackage{mathrsfs}
\usepackage{xcolor}
\usepackage{graphicx}
\usepackage{listings}
\usepackage{hyperref}
\usepackage{enumitem}

\hypersetup{%
  colorlinks=true,
  linkcolor=blue,
  linkbordercolor={0 0 1}
}
 
\renewcommand\lstlistingname{Algorithm}
\renewcommand\lstlistlistingname{Algorithms}
\def\lstlistingautorefname{Alg.}

\lstdefinestyle{Python}{
    language        = Python,
    frame           = lines, 
    basicstyle      = \footnotesize,
    keywordstyle    = \color{blue},
    stringstyle     = \color{green},
    commentstyle    = \color{red}\ttfamily
}

\setlength{\parindent}{0.0in}
\setlength{\parskip}{0.05in}

% Edit these as appropriate
\newcommand\course{10 - 725}
\newcommand\hwnumber{1}                  % <-- homework number
\newcommand\NetIDa{abc}           % <-- NetID of person #1

\pagestyle{fancyplain}
\headheight 35pt
\lhead{\NetIDa}
\lhead{\NetIDa}                 % <-- Comment this line out for problem sets (make sure you are person #1)
\chead{\textbf{\Large Homework \hwnumber}}
\rhead{\course \\ \today}
\lfoot{}
\cfoot{}
\rfoot{\small\thepage}
\headsep 1.5em


% if you want to create a new list from scratch
\newlist{alphalist}{enumerate}{1}
% in that case, at least label must be specified using \setlist
\setlist[alphalist,1]{label=\textbf{\alph*.}}

\begin{document}

\section*{Problem 1}

Answer to the problem goes here.

\begin{alphalist}

%==================================
\item Closed and convex sets.
\begin{enumerate}[label=(\roman*)]
    \item
        For any $Ax, Ay \in A(S)$ ($x, y \in S$) and any $\theta \in [0, 1]$, we have

        \[\theta Ax + (1 - \theta) Ay = A (\theta x + (1- \theta)y)\]

        is a member of $A(S)$ because $\theta x + (1-\theta) y \in S$.
        Therefore $A(S)$ is convex.

    \item
        For any $x, y \in A^{-1}(S)$, we have $Ax, Ay \in S$.

        Then $\theta Ax + (1-\theta)Ay = A(\theta x + (1-\theta)y) \in S,
        \forall \theta \in [0, 1]$ because $S$ is convex.

        By definition of $A^{-1}(S)$, $\theta x + (1-\theta)y \in A^{-1}(S)$. Therefore $A^{-1}(S)$ is convex.

    \item
        Let $x$ is a limit point of $A^{-1}(S)$, then $\forall{}\epsilon{}>0,
        \exists{}y \in A^{-1}(S): ||x - y|| = ||z|| < \epsilon$. We will show
        that $Ax$ is a limit point of $S$.

        Consider the l2-norm
        \[
        \begin{split}
            ||Ax - Ay|| &= ||Az||  = \left|\left|\begin{matrix} \sum_j A_{1j}z_j \\ \sum_j A_{2j}z_j \\ .. \\ \sum_j A_{mj}z_j\end{matrix}\right|\right|
                        = \sqrt{\sum_i^m\sum_j^nA_{ij}^2 z_j^2} \\
                        &\le \sqrt{\sum_i^m \sum_j^n A_{max}^2 z_j^2} =\sqrt{m} A_{max} ||z|| < \sqrt{m}A_{max}\epsilon \\
                        &\text{with} \quad A_{max} = \max{|A_{ij}|}.
        \end{split}
    \]

        So with point $Ax$, for all $\xi = \sqrt{m}A_{max}\epsilon>0$, we can find
        point $Ay \in S$ (because $y \in A^{-1}(S)$) such that $|| Ax - Ay || < \xi$. By definition, $Ax$
        is a limit point of $S$. Since $S$ is closed set, $Ax$ is also in $S$,
        it follow that $x \in A^{-1}(S)$.

        We have shown that any limit point $x$ of $A^{-1}(S)$ is also an
        element of $A^{-1}(S)$, therefore it is a closed set.

    \item Consider $S = \{(x, y) : x > 0, y \ge \frac{1}{x}\}$, S is a closed
        set because the compliment set $R^2 \backslash S$ equal to its interior (for any
        point, we can find a disc containing the point that doesn't intersect
        $S$).

        Image of S under transformation $A = (\begin{smallmatrix} 1 & 0
        \end{smallmatrix})$ is $A(S) = \{ x : x > 0 \}$, obviously is not a
        closed set.

\end{enumerate} % end of 1a

%==================================
\item Polyhedra.

\begin{enumerate}[label=(\roman*)]
    \item

    \item Suppose $P = \{x : Cx \le d\}$. Since $Q = \{ (x, y) : Cx \le d,
        y = Ax \}$ is a polyhedron, then $A(P) = \{y : (x, y) \in Q\}$ is also a
        polyhedron.
\end{enumerate}

%==================================
\item Let $X, Y$ are elements of the set. For each $\theta \in [0, 1]$ we have
    \begin{itemize}
        \item $\theta X + (1 - \theta)Y$ is symetric,
        \item $\theta X + (1 - \theta) \succeq 0$,
        \item $I - \theta X + (1 - \theta) = \theta(I - X) + (1-\theta)(I - Y) \succeq 0$,
        \item $tr(\theta X + (1- \theta)Y) = \theta tr(X) + (1 - \theta)tr(Y) = k$
    \end{itemize}

    so $\theta X + (1- \theta)Y$ is also an element. Therefore the set is convex.


%==================================
\item
    Suppose both statements
    \begin{itemize}
        \item there exists $x \in \mathbb{R}^n$ such that $Ax = b, x \ge 0$; (i)
        \item there exists $y \in \mathbb{R}^m$ such that $A^Ty \ge 0, y^Tb <
            0$; (ii)
    \end{itemize}

    are true. Then $y^Tb = x^TA^Tb < 0$. But $x \ge 0, A^Ty \ge 0 \implies
    x^TA^Ty \ge 0$. Contradiction.

    Consider two sets $C = \{t : t = Ax, x \ge 0\}, D = \{ b \}$. Then either
    statement (i) or "set C, D are disjoint" (iii) is true. We will show
    that (iii) implies (ii), so that exactly one of (i) or (ii) is true.

    We have $C, D$ are both convex, closed and disjoint and $D$ is bounded. By
    strict variant of the Separating Hyperplane Theorem, $C$ and $D$ can be
    separated; i.e.,
    \[
        \exists y \in \mathbb{R}^m, y \ne 0, z \in \mathbb{R} \quad \text{such
        that}\quad y^Tt \ge z, \forall t \in C \quad \text{and} \quad y^Tb < z
    \]

    Since $0 \in C$, we must have $z \le 0$. And if $\exists t \in C \quad
    \text{s.t.} \quad y^Tt < 0$ then $y^T\alpha{}t$ can be arbitrarily large negative
    as $\alpha \to \infty$. So
    \[
        y^Tt \ge 0, \forall t \in C  \qquad \text{and} \qquad y^Tb < 0
    \]
    Let $a_1, a_2,..., a_n$ be the columns of $A$, then $a_i \in C$. This
    implies $A^Ty \ge 0$ (statement ii).
\end{alphalist}

\newpage
\section*{Problem 2}
% Rest of the work...

\end{document}
